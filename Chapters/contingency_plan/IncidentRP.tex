\section{Incident response plan}

All attack scenarios dictated, by the ”subordinate plan classification addendum to end case”, to create an incident response plan is added to this chapter. The template used for all incident response plans is taken from the ”Principles of Incident Response and Disaster Recovery”. This details each action to be taken during an incident, when the incident is declared over, what actions to be taken when the incident is over, when the restoration process is over and which actions is to be taken in advance to an incident.

In addition to the incident response plan it is added a shortened incident response plan which describes the notification response corresponding to the incident. This describes how the person which discovered the attack and which action they should take when notifying the correct personnel.

\begin{longtable}{| p{4cm} | p{8cm} |}
	\hline \multicolumn{2}{| c |}{\textbf{Fire}}\\\hline
	\hline \multicolumn{2}{| c |}{\textbf{Incident response plan}}
	\endfirsthead
	
	\hline \multicolumn{2}{| c |}{\textbf{Fire}...continued}\\\hline
	\hline \multicolumn{2}{| c |}{Incident response plan}
	\endhead
	
	\multicolumn{2}{|r|}{\textit{continued on next page}}\\\hline
	\endfoot
	
	\endlastfoot
	
	\textbf{Attack type:} & Fire \\\hline
	\textbf{Trigger:} & 
	\begin{description}
	\item Strong drought
	\item Warning from Local authorities
	\item
	\end{description} \\\hline
	
	\textbf{Reaction force and lead:} & IT-manager, back-up CEO \\\hline
	\textbf{Notification method:} & Verbal communication by telephone or VoIP, physical interaction or physical. \\\hline
	\textbf{Response time:} & Maximum 12 hours from detection to notifying to the incident response task force and a maximum of 24 hours until countermeasures are putted in place or incident is discarded. \\\hline
	\multicolumn{2}{| c |}{\textbf{Actions to be taken during this response:}}\\\hline
	\multicolumn{2}{| c |}{
	\begin{enum}
		\item{:} Notify the IT-manager by phone, VoIP, email, physical interaction or physically by inside post service.
		\item{:} Preliminary assessment by the IT-manager of the situation.
		\item{:} Decision taking, by the IT-manager, to elevate the initial warning to the state of incident or to discard it.
		\item{:} Inform the CEO about the decision taken.
	\end{enum}
	}\\\hline
	\multicolumn{2}{| c |}{\textbf{Incident ends when:}}\\\hline
	\multicolumn{2}{| c |}{
	\begin{itemize}
		\item The incident is discarded
		\item The responsibility is shifted to the IR team
	\end{itemize}
	}\\\hline
	
	\multicolumn{2}{| c |}{\textbf{Preparation actions to be integrated into IR plans before incident response plan is needed:}}\\\hline
	\multicolumn{2}{| c |}{
	\begin{enum}
		\item Training the IT-managers to correctly assess a fire scenario
		\item Exercise routines for the call roster
	\end{enum}
	}\\\hline
	
	
	
	
	
	
	
	
	
	
	
	
	
					
\end{longtable}