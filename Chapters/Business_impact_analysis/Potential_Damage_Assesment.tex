\section{Potential Damage Assessment}

In this section is described, in great detail, the scenarios introduced in section \ref{sec:Successful_Attack}. For this purpose we used as template the example in the book "Principles of incident response and disaster recovery" \cite {whitman5}. The two different scenarios are detailed in four steps each. A detailed description of the attack and the countermeasures taken is provided. This description is repeated for the Best case, worst case and most probable scenarios.

\begin{longtable}{| p{4cm} | p{8cm} |}
	\hline \multicolumn{2}{| c |}{\textbf{Fire: Addendum}}\\\hline
	\endfirsthead
	
	\hline \multicolumn{2}{| c |}{\textbf{Fire: Addendum} ...continued}\\\hline
	\endhead
	
	\multicolumn{2}{|r|}{\textit{continued on next page}}\\\hline
	\endfoot
	
	\endlastfoot
	
	Date of analysis: & 15.07.2013 \\\hline
	
	Attack name/description: &  Fire in the Smoking Games headquarters.\\\hline
	
	Comments: & None at this time.\\\hline
	
	\multicolumn{2}{| c |}{\textbf{Best case scenario for this attack}}\\\hline
	
	\multicolumn{2}{| p{12cm} |}{
		\paragraph{Description:} The best case scenario it's when the fire doesn't produce damage to the building or any kind of asset. This can happen because the fire doesn't get to the building if produced outside or the fire extinguishing system works properly and it doesn't extend.
		\paragraph{Scenario risk:} Low, with the training of the employees for those situations. Also a good fire extinguishing system can prevent the extension.
		\paragraph{Scenario cost to organization:} moderate, the training of the employees can be done quite cheap. In the other hand the fire extinguishing system maintenance price it's quite expensive.
		\paragraph{Scenario probability of attack continuing and spreading:}none, once it's extinguished the threat is over.
	}\\\hline
	
	\multicolumn{2}{| c |}{\textbf{Worst case scenario for this attack}}\\\hline
	
	\multicolumn{2}{| p{12cm} |}{
		\paragraph{Description:} In this case the fire spreads all over the building making impossible to work there. It can also mean the loss of human assets, which is much worse.
		\paragraph{Scenario risk:}High, the hole enterprise must change the location. During that period the enterprise is vulnerable to any other kind of attack or threat. 
		\paragraph{Scenario cost to organization:} Very high, the reparation of the building and the necessity to use a different facilities during all the reparation time is a big cost for the enterprise.
		\paragraph{Scenario probability of attack continuing and spreading:}None, even though the big cost the fire cannot continue once it's extinguished.
	}\\\hline
	
	\multicolumn{2}{| c |}{\textbf{Most likely case scenario for this attack}}\\\hline
	
	\multicolumn{2}{| p{12cm} |}{
		\paragraph{Description:} Minor fire that doesn't affect the hole structure of the building. It can cause minor disturbances in the business processes. It will not disturb the full enterprise procedure.
		\paragraph{Scenario risk:} Medium, the department affected by the fire will not be able to operate for certain time and should be relocated. The risk for human beings will continue to be low.
		\paragraph{Scenario cost to organization:}High, the rehabilitation of the building will be a high price. Also the training of the  employees includes an extra cost for the enterprise.
		\paragraph{Scenario probability of attack continuing and spreading:}None, even though the cost the fire cannot continue once it's extinguished.
	}\\\hline
	
\end{longtable}

\begin{longtable}{| p{4cm} | p{8cm} |}
	\hline \multicolumn{2}{| c |}{\textbf{Denial of Service Attack: Addendum}}\\\hline
	\endfirsthead
	
	\hline \multicolumn{2}{| c |}{\textbf{Denial of Service Attack: Addendum} ...continued}\\\hline
	\endhead
	
	\multicolumn{2}{|r|}{\textit{continued on next page}}\\\hline
	\endfoot
	
	\endlastfoot
	
	Date of analysis: & 30.12.2013 \\\hline
	
	Attack name/description: &  Denial of service attack against the web services of Smoking games.\\\hline
	
	Comments: & None at this time.\\\hline
	
	\multicolumn{2}{| c |}{\textbf{Best case scenario for this attack}}\\\hline
	
	\multicolumn{2}{| p{12cm} |}{
		\paragraph{Description:} The best case scenario it's when the attack doesn't get to affect the enterprise servers. 
		\paragraph{Scenario risk:} Medium, the number of this attacks is quite high. At the same time the probability of a successful scenario still being low. 
		\paragraph{Scenario cost to organization:} Low, Without servers going down the enterprise doesn't lose money. To put the servers up and running again should not cost too much neither.
		\paragraph{Scenario probability of attack continuing and spreading:}none, if the servers don't go down There is low probability of more attacks afterwards.
	}\\\hline
	
	\multicolumn{2}{| c |}{\textbf{Worst case scenario for this attack}}\\\hline
	
	\multicolumn{2}{| p{12cm} |}{
		\paragraph{Description:} A successful DDoS attack is launched against Smoking games. This attack affects the public servers and the routers. It can also give an entrance to more experienced hackers to get an access to the private network of the enterprise.
		\paragraph{Scenario risk:}Low, the current software and hardware makes an attack of this kind difficult. It will need a very experienced hacker or hacker group to take it to an end.
		\paragraph{Scenario cost to organization:} High, to lose the frontend of the enterprise can mean loosing a lot of customers and profits. Also the reparation or substitution of the affected hardware can be quite costly.
		\paragraph{Scenario probability of attack continuing and spreading:}Very high, this attack will not continue on itself but it will give an opening to new attacks.
	}\\\hline
	
	\multicolumn{2}{| c |}{\textbf{Most likely case scenario for this attack}}\\\hline
	
	\multicolumn{2}{| p{12cm} |}{
		\paragraph{Description:} Slow servers or nearly crashing. This will affect the online services of the enterprise, but it doesn't mean that all of them will be broken.
		\paragraph{Scenario risk:} Medium, the risk of the attack on itself is low but it can be augmented if more attackers realize that a previous attack is working.
		\paragraph{Scenario cost to organization:}Medium, the hardware will have only minor affection in the attack. To update the software and look for bugs will be more costly. This added to the possible loss due to the laggy user experience can make quite a cost for the enterprise.
		\paragraph{Scenario probability of attack continuing and spreading:}Medium, as said before, in case more attackers can notice a first successful attack they could launch more attacks against the enterprise.
	}\\\hline
	
\end{longtable}